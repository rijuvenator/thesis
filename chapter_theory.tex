\newcommand{\mtworow}[1]{\multirow{2}{*}{#1}}

\chapter{The Standard Model and Beyond}
\label{chap:theory}
The Standard Model (SM) is the theory describing the electromagnetic, weak, and strong fundamental forces, as well as the kinematics of and interactions between all known elementary particles.
\Tabs~\ref{tab:sm:fermions}--\ref{tab:sm:bosons} enumerate the fundamental fermions and bosons of the SM, along with their quantum numbers and gauge symmetries.

\begin{table}
  \centering
  \begin{tabular}{cccc|ccccc}
    \hline 
    & \multicolumn{3}{c|}{Generation} & \mtworow{$C$} & \mtworow{$T$} & \mtworow{$T_3$} & \mtworow{$Y$} & \mtworow{$Q$}\\
    & I & II & III & & & & &\\
    \hline
    & & & & & & & &\\
    \multirow{4}{*}{Quarks} &
    \mtworow{$\begin{pmatrix*}[c]\;u\;\\d'\end{pmatrix*}_L$} &
    \mtworow{$\begin{pmatrix*}[c]\;c\;\\s'\end{pmatrix*}_L$} &
    \mtworow{$\begin{pmatrix*}[c]\;t\;\\b'\end{pmatrix*}_L$} &
            \mtworow{r, g, b} & \mtworow{$1/2$} & $+1/2$      & \mtworow{$+1/3$} & $+2/3$\\
    & & & &                   &                 & $-1/2$      &                  & $-1/3$\\

                     &
    \mtworow{$\begin{matrix*}[c]\;u_R\;\\d_R\end{matrix*}$} &
    \mtworow{$\begin{matrix*}[c]\;c_R\;\\s_R\end{matrix*}$} &
    \mtworow{$\begin{matrix*}[c]\;t_R\;\\b_R\end{matrix*}$} &
            \mtworow{r, g, b} & \mtworow{0}     & \mtworow{0} &          $+4/3$  & $+2/3$\\
    & & & &                   &                 &             &          $-1/3$  & $-1/3$\\

    & & & & & & & &\\

    \multirow{4}{*}{Leptons} &
    \mtworow{$\begin{pmatrix*}[c]\nu_e   \\e   \end{pmatrix*}_L$} &
    \mtworow{$\begin{pmatrix*}[c]\nu_\mu \\\mu \end{pmatrix*}_L$} &
    \mtworow{$\begin{pmatrix*}[c]\nu_\tau\\\tau\end{pmatrix*}_L$} & 
             \mtworow{0}      & \mtworow{$1/2$} & $+1/2$      & \mtworow{$-1  $} & $0$  \\
    & & & &                   &                 & $-1/2$      &                  & $-1$ \\

                      &
    \mtworow{$\begin{matrix*}[c]\nu_{e R}   \\e_R   \end{matrix*}$} &
    \mtworow{$\begin{matrix*}[c]\nu_{\mu R} \\\mu_R \end{matrix*}$} &
    \mtworow{$\begin{matrix*}[c]\nu_{\tau R}\\\tau_R\end{matrix*}$} &
             \mtworow{0}      & \mtworow{0}     & \mtworow{0} &          $0   $  & $0$  \\
    & & & &                   &                 &             &          $-2  $  & $-1$ \\
  \end{tabular}
  \caption{Fermion (spin-1/2) content of the Standard Model.}
  \label{tab:sm:fermions}
\end{table}

\begin{table}
  \centering
  \begin{tabular}{ccccc}
    \hline
    Name & Symbol & Gauge Group & Number \\
    \hline
    Gluon & $g$ & $\mathrm{SU}(3)$ & 8 \\
    W & $W^\pm$ & $\mathrm{SU}(2)$ & 2 \\
    Z & $Z$ & $\mathrm{SU}(2)\times\mathrm{U}(1)$ & 1 \\
    Photon & $\gamma$ & $\mathrm{U}(1)$ & 1 \\
    Higgs & $H$ & & 1 \\
    \hline
  \end{tabular}
  \caption{Boson content of the Standard Model.}
  \label{tab:sm:bosons}
\end{table}

The SM is (mathematically) formalized as a gauge quantum field theory described by a Lagrangian density composed of quantum fields.
The defining characteristic of this Lagrangian is its invariance under local transformations of the fields under the gauge group
\begin{equation}
  \mathrm{SU}(3)\times\mathrm{SU}(2)\times\mathrm{U}(1)
  \label{eq:sm:gaugegroup}
\end{equation}
By virtue of Noether's theorem, continuous symmetries of the Lagrangian correspond to conserved currents; here, $\mathrm{SU}(3)$ corresponds to conservation of color charge; $\mathrm{SU}(2)$ corresponds to weak isospin; and $\mathrm{U}(1)$ corresponds to weak hypercharge.

\subsection{Notation of the SM Lagrangian}
A Lagrangian can be generally described in three pieces: kinetic terms involving derivatives, mass terms that are second-order in the fields, and interaction terms that are third-order and higher in the fields describing interactions between fields.
The kinetic and mass terms together are called the free Lagrangian, since they describe a free field not interacting with anything else.

The free Lagrangian for a spin-1/2 spinor fermion field $\psi$ of mass $m$ is described by the Dirac Lagrangian:
\begin{equation}
  \mathcal{L}_\text{Dirac} = i\overline{\psi}\gamma^\mu\partial_\mu\psi - m\overline{\psi}\psi
  \label{eq:sm:dirac}
\end{equation}

For a spin-1 vector boson field $A_\mu^i$ associated with a gauge group with gauge coupling $g$, define the field strength tensor
\begin{equation}
  F_{\mu\nu}^i = \partial_\mu A_\nu^i - \partial_\nu A_\mu^i + g f^{ijk} A_\mu^j A_\nu^k
  \label{eq:sm:fieldstrength}
\end{equation}
where the structure constants $f^{ijk}$ are defined by commutators of the group generators $T_i$:
\begin{equation}
  [T_i, T_j] = if^{ijk}T_k
  \label{eq:sm:structureconstants}
\end{equation}
where $i$ labels the group generators.
For Abelian groups (such as $\mathrm{U}(1)$), the group generators (for $\mathrm{U}(1)$, complex phases of modulus 1) commute, and the structure constants vanish.
For non-Abelian groups (such as $\mathrm{SU}(3)$ and $\mathrm{SU}(2)$), the group generators (for $\mathrm{SU}(2)$, the Pauli matrices; for $\mathrm{SU}(3)$, the Gell-Mann matrices) do not commute.
The free Lagrangian for a spin-1 vector boson field $A_\mu$ of mass $m$ is described by the Proca Lagrangian:
\begin{equation}
  \mathcal{L}_\text{Proca} = -\frac14 F_{\mu\nu}F^{\mu\nu} + \frac12m^2A_{\nu}A^\nu
  \label{eq:sm:proca}
\end{equation}

The free Lagrangian for a spin-0 scalar boson field $\phi$ of mass $m$ is described by the Klein-Gordon Lagrangian:
\begin{equation}
  \mathcal{L}_\text{Klein-Gordon} = (\partial_\mu\phi)^2 - m^2 \phi^2
  \label{eq:sm:kg}
\end{equation}

Each fermion field $\psi$ can be decomposed into left and right-handed chiral components:
\begin{equation}
  \psi_L = \frac12\left(1-\gamma^\mu\right)\psi,\;\psi_R = \frac12\left(1+\gamma^\mu\right)\psi
  \label{eq:sm:chiral}
\end{equation}

The SM Lagrangian begins with kinetic Proca terms for
\begin{itemize}
  \item 1 $\mathrm{U}(1)$ gauge boson $B_\mu$
  \item 3 $\mathrm{SU}(2)$ gauge bosons $W_\mu^i$, $i = 1, 2, 3$
  \item 8 $\mathrm{SU}(3)$ gauge bosons $G_\mu^i$, $i = 1 \ldots 8$
\end{itemize}
as well as kinetic Dirac terms for three ``generations'' of fermions, each consisting of
\begin{itemize}
  \item 1 charged lepton $\ell$ with electric charge $-1$
  \item 1 neutrino $\nu_\ell$ with electric charge 0
  \item 3 up-type quarks $u_i$ with electric charge $+2/3$ and color charges $i = r, g, b$
  \item 3 down-type quarks $d_i$ with electric charge $-1/3$ and color charges $i = r, g, b$
\end{itemize}
and their antiparticles.
All bosons and fermions begin initially massless.
For convenience in the forthcoming description of their interactions with gauge bosons, all fermions are decomposed into their left and right-handed chiral components.
The free SM Lagrangian thus has 12 terms of the form $-\frac14 F_{\mu\nu}F^{\mu\nu}$, and 48 terms of the form $i\overline{\psi}\gamma^\mu\partial_\mu\psi$, plus their Hermitian conjugates for the antiparticles, where each $\psi$ is two-component Weyl spinor.

If the fields corresponding to the three colors of each quark are collected into a three-component vector,
\begin{equation}
  \psi_\text{color} = \begin{pmatrix}\psi_r\\\psi_g\\\psi_b\end{pmatrix}
  \label{eq:sm:color}
\end{equation}
then three terms of the Dirac Lagrangian can be written $i\overline{\psi_\text{color}}\gamma^\mu\partial_\mu\psi_\text{color}$.
Such a term is invariant under a global transformation
\begin{equation}
  \psi \to \psi' = U\psi,\; U = e^{i\mathbf{T}\cdot\bm{\theta}} \in G
  \label{eq:sm:global}
\end{equation}
where as before $\mathbf{T} = T_i$ are the group generators of the group $G$, and here $G = \mathrm{SU}(3)$.
The Lagrangian is thus said to be invariant under global $\mathrm{SU}(3)$ transformations.
Similarly, if the fields corresponding to left-handed charged leptons and neutrinos, and left-handed up-type and down-type quarks, are collected into two-component vectors,
\begin{equation}
  \psi_{\ell\,\text{flavor}} = \begin{pmatrix}\nu_\ell\\\ell\end{pmatrix}_L,\; \psi_{q\,\text{flavor}} = \begin{pmatrix}u\\d\end{pmatrix}_L
  \label{eq:sm:su2}
\end{equation}
then terms of the Dirac Lagrangian can be collected as above, and such terms are invariant under global $\mathrm{SU}(2)$ transformations.

\begin{itemize}
  \item The gluons interact with fermions with nonzero color charge $C$, i.e. quarks
  \item The W bosons interact with fermions with nonzero weak isospin $T$, i.e. the left-handed chiral components of all fermions
  \item The B boson interacts with fermions with nonzero weak hypercharge $Y$, i.e. all fermions except the right-handed chiral components of neutrinos
\end{itemize}

These interactions can be generated from the free SM Lagrangian by requiring it be invariant under local gauge transformations.
The Lagrangian must be modified in a way so that the global transformation of \Eq~\ref{eq:sm:global} can hold locally:
\begin{equation}
  \psi \to \psi' = U\psi,\; U = e^{i\mathbf{T}\cdot\bm{\theta}(x)} \in G
  \label{eq:sm:local}
\end{equation}
This is achieved by replacing all derivatives $\partial_\mu$ in the fermion kinetic terms with an appropriate ``covariant derivative'' $\mathcal{D}_\mu$, generating the interaction terms between fermions and gauge bosons.

All SM fermions and the W and Z gauge bosons are observed to have a mass, but the previous treatment assumes all particles to be massless.
Adding appropriate mass terms would result in the Lagrangian losing its local gauge invariance.
In the SM, particles acquire mass terms by interacting with a massive scalar field (the spin-0 Higgs boson H), whose potential exhibits a nonzero vacuum expectation value.
This phenomenon is known as spontaneous symmetry breaking, and the mechanism by which it is accomplished is known as the Higgs mechanism.
After symmetry breaking, the B and W bosons mix to form the mass eigenstates that are experimentally observed: the W and Z bosons.

\subsection{Fermi's Golden Rule}
A fundamental result of quantum field theory is the relativistic version of Fermi's Golden Rule.
Suppose particle X of mass $M$ decays into two muons. Then the decay rate $\Gamma$ of X is given by
\begin{equation}
  \Gamma_X = \frac{|\mathbf{p}|}{16\pi M^2}\left|\mathcal{M}\right|^2 
  \label{eq:sm:fermi}
\end{equation}
where 
\begin{equation}
  |\mathbf{p}| = \frac{1}{2M}\sqrt{M^4 - 4M^2m_\mu^2}
  \label{eq:sm:p}
\end{equation}
Lagrangian such as the one describing the SM outlined in the previous section prescribes a set of Feynman rules
Section that mentions that Lagrangians prescribe Feynman rules, the result of which are terms for cross sections and lifetimes which are experimentally determinable.

Discuss Fermi's golden rule, involving matrix element (depending on gauge couplings) and phase space factor (depending on mass splittings).

\subsection{Beyond the SM}
Section mentioning some of the motivations for extending the standard model: neutrino oscillations, hierarchy, gravity.

Discuss how short lifetimes were not found at the LHC.

Discuss motivations for long lifetimes: small mass splittings and small gauge couplings.

Discuss benchmark models with free parameters that are not part of the SM.
