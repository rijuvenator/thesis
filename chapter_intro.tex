\chapter{Introduction}
\PretentiousQuote{Quantum field theory arose out of our need to describe the ephemeral nature of life.}{A. Zee}{Quantum Field Theory in a Nutshell}

Elementary particle physics is a field of scientific inquiry that attempts to answer one of humanity's deepest and most fundamental questions: what are we made of?
The unifying power of its underlying theory has been nothing short of spectacular, describing three of the four fundamental forces of nature and all known fundamental particles in a single breath of elegant mathematics.
This theory is known as the Standard Model, one of the crown jewels of twentieth century particle physics, and so successful has it been in fact that to date, nearly all measurements performed by four decades of ever increasingly sophisticated experiments have been harmonious with its predictions.

Nevertheless, the Standard Model is believed to be incomplete for a number of reasons, ranging from unfulfilled aesthetic guiding principles to unexplained experimentally observed phenomena such as gravity, dark matter, and neutrino mass.
A renowned achievement of the field is the discovery of the Higgs boson in 2012, and its subsequent confirmation (as predicted by the Standard Model) as the particle formed from interactions giving all known particles their mass.
Yet since then, no new particles, heralds of an exciting landscape of unexplored phenomena, have been confirmed.
The quest continues.

High-energy proton-proton (\pp) collisions surrounded by a network of particle detectors provide a lens with which to study the Standard Model and to search for hints of what lies beyond.
This thesis presents a search for new particles with long lifetimes, in a parameter space not yet fully explored by the large, general-purpose experiments based at the Large Hadron Collider (LHC) at CERN.
The analysis is performed using Compact Muon Solenoid (CMS) data taken during Run 2 of the LHC, and is heir to two CMS analyses \cite{EXO-12-037,CMS-PAS-EXO-14-012} performed with data taken during Run 1 of the LHC.
No analysis searching for a faint hint of new physics over a sea of background effects could be performed without a keen understanding of the behavior of the experimental apparatus, and so this thesis also presents a contribution to that understanding: a measurement of background hits induced by neutrons in the cathode strip chamber muon detectors found in the endcaps of CMS, with data taken both by the CMS experiment and at the CERN Gamma Irradiation Facility (GIF++) located near the CERN Super Proton Synchrotron.
Potential implications of this background for the performance of the High-Luminosity LHC (HL-LHC) are also discussed.

The thesis is organized as follows.
Chapter~\ref{chap:theory} presents theoretical motivations for the existence of these particles, as well as arguing for the discovery sensitivity of these experiments.
Chapter~\ref{chap:cms} is an overview of the CMS experimental apparatus within the accelerator complex of the LHC.
Chapter~\ref{chap:displaced} describes the search for new long-lived particles decaying to two muons, using CMS data taken in 2016 at a center-of-mass energy of 13\TeV corresponding to 36.3\fbinv of integrated luminosity.
This analysis searches for vertices characteristically displaced from the proton-proton collision point, signalling the potential decay of a long-lived particle.
Chapter~\ref{chap:conclusion} summarizes the analysis and proposes extensions to it for further study.
Appendix~\ref{chap:neutron} presents the study of neutron-induced background hits in the endcap muon system of CMS, and studies performed with muon test beam data taken by similar muon chambers at GIF++.
