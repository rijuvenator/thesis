\chapter{Summary and Concluding Remarks}
\label{chap:conclusion}
\PretentiousQuoteWithAuthor{One sees clearly only with the heart. Anything essential is invisible to the eyes.}{The Fox}{The Little Prince}{Antoine de Saint-Exup\'{e}ry}

This thesis presented a search for new long-lived particles decaying to two muons in the CMS detector with \pp collision data taken in 2016 at $\sqrt{s} = 13\TeV$ corresponding to 36.3\fbinv of integrated luminosity during Run~2 of the LHC.
The search presented in this thesis used only the muon system in order to probe the longest lifetimes to which the LHC experiments are sensitive.
The results were interpreted in terms of a benchmark model consisting of BSM Higgs bosons decaying to long-lived scalar bosons, but were presented in an inclusive and approximately model-independent way.
No excess over the expected Standard Model background was observed.

A notable improvement to the analyses performed with Run~1 data \cite{EXO-12-037,CMS-PAS-EXO-14-012} is a higher efficiency for selecting signal events, primarily a result of the development of the \DSAToPAT association procedure designed to maintain this efficiency and of the use of the DSA muon reconstruction using cosmic muon seeds over the previously used RSA muon reconstruction.
Furthermore, estimating background events with two independent categories---Drell-Yan events and QCD events---allowed the analysis to remain sensitive to long-lived particle masses in a region with significant QCD background.
The resulting upper limits are consistent with or improved compared to the analyses performed with Run~1 data.

Along with the results of this thesis work, several potential improvements to the analysis using dimuons formed from two DSA muons have been identified that may be implemented by other CMS researchers in the near future.
First, the analysis will benefit from the addition of data taken at the LHC in 2018, not only from the increased sample size but also from an improved, dedicated displaced dimuon trigger implemented for 2018 data taking.
Second, parametrizing certain selections as a function of long-lived particle lifetime, such as the \LxySig selection, can further discriminate signal events from background events, especially at long lifetimes.
Third, technical improvements to the \DSAToPAT association procedure involving the segments associated to global-only PAT muons will fine-tune the association and reduce a population of background events.
And finally, a technical implementation of the muon timing information for DSA muons, rather than use of the timing from a standalone muon spatially nearby, will further reject out-of-time background events as well as provide an additional method for suppressing cosmic muon events.

In addition, the companion analyses using PAT muons and corresponding dimuons formed from two PAT muons or a PAT muon and a DSA muon will cover the entire sensitivity of the detector.
The improved spatial resolution given by the silicon tracker will translate into more efficient signal discrimination and background rejection for long-lived particle decays at a few centimeters.

The exploration of the landscape of new potential long-lived exotic particles is an exciting one, with rapid development in recent years of new analysis techniques to tackle previously unforeseen experimental challenges.
This thesis presented a contribution to that exploration, with the hope of illuminating paths to new horizons yet to be undertaken by further studies.
